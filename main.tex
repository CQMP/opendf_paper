%% The '3p' and 'times' class options of elsarticle are used for Elsevier CRC
%% The 'procedia' option causes ecrc to approximate to the Word template
\documentclass[3p,times,procedia]{elsarticle}
\flushbottom
\bibliographystyle{elsarticle-num}

%% The `ecrc' package must be called to make the CRC functionality available
\usepackage{ecrc}
\usepackage[colorlinks=true]{hyperref}%url}
\usepackage{amsmath}
\usepackage{epstopdf}
\usepackage{todonotes}
\usepackage{color}
 \usepackage{lineno}

\newcommand{\egtodo}[2]{\todo[inline,backgroundcolor=orange!20!white]{\textcolor{cyan}{{\bf \it by: #1}:  {\bf Emanuel:} #2}}}
\newcommand{\jltodo}[2]{\todo[inline,backgroundcolor=orange!20!white]{\color{Maroon}{{\bf \it by: #1}: {\bf James:} #2}}}
\newcommand{\aatodo}[2]{\todo[inline,backgroundcolor=orange!20!white]{\color{red}{{\bf \it by: #1}: {\bf Andrey:}  {\huge BLERG!!!!}: #2}}}

%% The ecrc package defines commands needed for running heads and logos.
%% For running heads, you can set the journal name, the volume, the starting page and the authors

%% set the volume if you know. Otherwise `00'
\volume{00}

%% set the starting page if not 1
\firstpage{1}

%% Give the name of the journal
\journalname{Physics Procedia}

%% Give the author list to appear in the running head
%% Example \runauth{C.V. Radhakrishnan et al.}
\runauth{A. E. Antipov et al.}

%% The choice of journal logo is determined by the \jid and \jnltitlelogo commands.
%% A user-supplied logo with the name <\jid>logo.pdf will be inserted if present.
%% e.g. if \jid{yspmi} the system will look for a file yspmilogo.pdf
%% Otherwise the content of \jnltitlelogo will be set between horizontal lines as a default logo

%% Give the abbreviation of the Journal.
\jid{phpro}

%% Give a short journal name for the dummy logo (if needed)
\jnltitlelogo{Physics Procedia}


\usepackage{amssymb}
%% The amsthm package provides extended theorem environments
%% \usepackage{amsthm}

%% The lineno packages adds line numbers. Start line numbering with
%% \begin{linenumbers}, end it with \end{linenumbers}. Or switch it on
%% for the whole article with \linenumbers after \end{frontmatter}.
%%
%% natbib.sty is loaded by default. However, natbib options can be
%% provided with \biboptions{...} command. Following options are
%% valid:

%%   round  -  round parentheses are used (default)
%%   square -  square brackets are used   [option]
%%   curly  -  curly braces are used      {option}
%%   angle  -  angle brackets are used    <option>
%%   semicolon  -  multiple citations separated by semi-colon
%%   colon  - same as semicolon, an earlier confusion
%%   comma  -  separated by comma
%%   numbers-  selects numerical citations
%%   super  -  numerical citations as superscripts
%%   sort   -  sorts multiple citations according to order in ref. list
%%   sort&compress   -  like sort, but also compresses numerical citations
%%   compress - compresses without sorting
%%
%\biboptions{authoryear}

% \biboptions{}

% if you have landscape tables
\usepackage[figuresright]{rotating}
%\usepackage{harvard}
% put your own definitions here:x
%   \newcommand{\cZ}{\cal{Z}}
%   \newtheorem{def}{Definition}[section]
%   ...

% add words to TeX's hyphenation exception list
%\hyphenation{author another created financial paper re-commend-ed Post-Script}

% declarations for front matter

\begin{document}

\begin{frontmatter}

%% Title, authors and addresses

%% use the tnoteref command within \title for footnotes;
%% use the tnotetext command for the associated footnote;
%% use the fnref command within \author or \address for footnotes;
%% use the fntext command for the associated footnote;
%% use the corref command within \author for corresponding author footnotes;
%% use the cortext command for the associated footnote;
%% use the ead command for the email address,
%% and the form \ead[url] for the home page:
%%
%% \title{Title\tnoteref{label1}}
%% \tnotetext[label1]{}
%% \author{Name\corref{cor1}\fnref{label2}}
%% \ead{email address}
%% \ead[url]{home page}
%% \fntext[label2]{}
%% \cortext[cor1]{}
%% \address{Address\fnref{label3}}
%% \fntext[label3]{}

\dochead{28th Annual CSP Workshop on ``Recent Developments in Computer Simulation Studies in Condensed Matter Physics'', CSP 2015}
%% Use \dochead if there is an article header, e.g. \dochead{Short communication}
%% \dochead can also be used to include a conference title, if directed by the editors
%% e.g. \dochead{17th International Conference on Dynamical Processes in Excited States of Solids}

\title{\texttt{opendf} - an implementation of the dual fermion method for strongly correlated systems }

%% use optional labels to link authors explicitly to addresses:
%% \author[label1,label2]{<author name>}
%% \address[label1]{<address>}
%% \address[label2]{<address>}



\author[a]{Andrey E. Antipov\corref{cor1}} 
\author[a]{James P.F. LeBlanc}
\author[a]{Emanuel Gull}

\address[a]{Department of Physics, University of Michigan, Ann Arbor, Michigan 48109, USA}

\begin{abstract}
The dual fermion method is a multiscale approach for solving lattice problems of interacting strongly correlated systems. In this paper, we present the \textbf{opendf} code, an open-source implementation of the dual fermion method applicable to Hubbard models in dimensions $D=1,2,3,4$. The method is built on a dynamical mean field starting point, which neglects all local correlations, and perturbatively adds spatial correlations to it. Our code is distributed as an open-source package under the Gnu public license version 2.
\end{abstract}

\begin{keyword}
Dual Fermions; Dynamical Mean Field Theory
\end{keyword}
\cortext[cor1]{Corresponding author.}
\end{frontmatter}

%\correspondingauthor[*]{Corresponding author. Tel.: +0-000-000-0000 ; fax: +0-000-000-0000.}
\email{aantipov@umich.edu}
\vspace*{-8pt}
\linenumbers

\listoftodos
\section{Introduction}

Understanding the physics of complex correlated electron systems beyond simple approximations or exactly solvable limits is a long-standing goal of condensed matter physics. The dynamical mean field theory (DMFT) \cite{MetznerVollhardt:1989,MH89,Georges92,Jarrell92,Georges96,KotliarSavrasov:2006} provides a workhorse for numerically simulating the physics of such systems: It establishes that, if correlations and interactions are assumed to be local, the (intractable) extended system can be mapped onto an  Anderson impurity model, which can then be solved numerically.

The dynamical mean field approximation of locality is often precise enough that general materials trends can be reproduced. Nevertheless, cases where the lack of momentum-dependent correlations leads to incorrect behavior are known \cite{SomePGReference}, and therefore methods that improve on the DMFT are needed. The Dual Fermion method \cite{Rubtsov2008}, which takes DMFT results as a starting point and perturbatively adds corrections to it, thereby reintroducing some momentum dependent correlations, is one of them. If all corrections are included, the method recovers the full momentum dependence of the original problem and becomes numerically exact.

In this paper, we present \texttt{opendf}, an implementation of the `ladder series' variant of the Dual Fermion method \cite{HafermannLi:2009}. This variant is  approximate, as neither vertices with more than two legs nor series of vertices beyond a single ladder are considered. Nevertheless, it has been  applied successfully to phase transitions \cite{HafermannLi:2009,Antipov2014,Li2014} \textcolor{red}{and probably other things and PT may not be the best application}.

Dual fermion calculations rely on a dynamical mean field input, which can be provided by one of the publicly available open source software packages that implement the approximation, including ALPS (the Algorithms and Libraries for Physics Simulations) \cite{ALPS2}, TRIQS (the Toolbox for Research on Interacting Quantum Systems) \cite{TRIQS}, and iQSIT\cite{iQIST}.
This initial step requires the solution of an interacting quantum many-body system and is computationally much more expensive than the summation of the dual fermion diagrams.

The rest of this paper is organized as follows: Section~\ref{sec:meth} introduces the methodology. Section~\ref{sec:imp} describes  implementation aspects, section~\ref{sec:perf} performance aspects, and section~\ref{sec:ex} shows some examples. Section \ref{sec:conclusions} has conclusions.

\section{Methodology}\label{sec:meth}
\subsection{Prerequisites}
Consider the Hamiltonian, $H$, for a lattice fermionic model in $D$ dimensions,
\begin{equation}
H = \sum_{k\sigma} (\varepsilon_k - \mu) c^\dagger_{k\sigma} c_{k\sigma} + \sum_i H^{\mathrm{int}} [c^\dagger_i, c_i],
\end{equation}
written in mixed momentum, $k$, and real space, $i$, notation in terms of creation and annihilation operators ($c^\dagger_{k\sigma}$ and $c_{k\sigma}$ respectively).  The index $\sigma$ labels the spin projection, $\varepsilon_k$ is the lattice dispersion relation and $k$ is the vector in the reciprocal space. 
$H^{\mathrm{int}}$ is the local interaction for each site, $i$, on the lattice. 
No assumption is made within DF as to the structure of $H^{\mathrm{int}}$.  For example, $H^{\mathrm{int}} = U c^\dagger_\uparrow c_\uparrow c^\dagger_\downarrow c_\downarrow$ would refer to the typical Hubbard model \cite{Hubbard1963}. 

As a first step, which must be performed outside of this code, an approximate solution of the model is obtained from a dynamical mean field calculation, for example provided by the ALPS code \cite{ALPS2} with an appropriate impurity solver \cite{Hafermann2013a}.
It provides an estimate for the local Green's function of the lattice problem as a  solution of the Anderson impurity model, embedded into a self-consistently determined hybridization. The imaginary time action of this ``impurity problem''  reads  
\begin{equation}
S^{\mathrm{A}} = -\sum_{i\omega,\sigma} (i\omega + \mu - \Delta_{\omega\sigma}) c^*_{\omega\sigma} c_{\omega\sigma} + S^{\mathrm{int}},
\end{equation}
where $S^{int} = \int_0^\beta d\tau H^{\mathrm{int}} [c^\dagger_i(\tau), c_i(\tau)] $ is the interaction part of the action and $\Delta_{\omega\sigma}$ is a self-consistently determined hybridization function. 
The one particle Green's function $g_{\omega\sigma} = -\langle c_{\omega\sigma} c^\dagger_{\omega\sigma} \rangle$ of the of the Anderson impurity model and the two particle vertex functions (i.e. the connected parts of two-particle Green's functions)  
\begin{equation}\label{eqn:vertex}
\gamma_{\Omega\omega\omega'}^{\sigma_1\sigma_2\sigma_3\sigma_4} = \left(\langle c_{\omega,\sigma_1} c^\dagger_{\Omega + \omega,\sigma_2} c_{\omega' + \Omega, \sigma_3} c^\dagger_{\omega', \sigma_4}\rangle - g_{\omega\sigma_1}g_{\omega'\sigma_3}\delta_{\Omega,0}\delta_{\sigma_1,\sigma_2} + g_{\omega\sigma_1} g_{\omega + \Omega, \sigma_2} \delta_{\omega,\omega'}\delta_{\sigma, \sigma_3} \right)
\end{equation}
are the input to DF calculations. 

The presented version of the code considers spin-symmetric solutions of the lattice $s=1/2$ fermionic problems, and does not describe symmetry-broken phases. We will omit the spin index $\sigma$ in single particle quantities in what follows, such that the complete set of input quantities reads
\begin{itemize}
\item $g_\omega$ - the full Green's function of the DMFT impurity problem (same values for both spin components)
\item $\Delta_{\omega}$ - hybridization function of the DMFT impurity problem
\item $\mu$ - chemical potential of the problem
\item Two independent components of the impurity vertex function, $\gamma_{\Omega\omega\omega'}^{\sigma_1\sigma_2\sigma_3\sigma_4}$, from Eqn~\eqref{eqn:vertex}: $\gamma_{\Omega,\omega,\omega'}^{\uparrow\uparrow\uparrow\uparrow} \equiv \gamma^{\uparrow\uparrow}_{\Omega,\omega,\omega'} $ and $\gamma_{\Omega,\omega,\omega'}^{\uparrow\downarrow\downarrow\uparrow} \equiv  \gamma^{\uparrow\downarrow}_{\Omega,\omega,\omega'} $
%\item Lattice dispersion $\varepsilon_{k}$ and the dimensionality $D$ are compiled into the specific running executable. The hypercubic lattice dispersion $-2t \sum_{i=1}^d \cos k_i$ at dimensions $D=1,2,3,4$ is provided with the code. Other lattice choices can be achieved by extending the code. 
\end{itemize}

\subsection{Ladder dual fermion self-consistency loop}
A precise derivation of the DF equations can be found in \cite{Hafermann2012, Antipov2014}, here we outline the equations solved within the code. The evaluation of the DF equations consists of iterative updates of the fully dressed dual fermion Green's function $\tilde G$, until convergence is achieved. It starts with the construction of the bare dual fermion propagator
\begin{equation}
\tilde G^{(0)}_{\omega,k} = \left[g_{\omega}^{-1} + \Delta_\omega - \varepsilon_k\right]^{-1} - g_{\omega}, \label{eq:gd0}
\end{equation}
which represents a $k$-dependent correction to the impurity Green's function. This Green's function is used to construct two-particle bubbles: 
\begin{equation}\label{eq:dual_bubble}
\tilde \chi_{\Omega\omega}(q) = -\frac{T}{N_k^D} \sum_k \tilde G_{\omega, k} \tilde G_{\omega + \Omega, k+q}.
\end{equation}
Here the integral over the Brilloin zone is replaced with a discrete summation with $N_k$ points in each direction. The impurity vertex functions are combined into density and magnetic channels (labeled d/m respectively) as: 
\begin{equation}\label{eq:spin_symm}
\gamma^{d/m}_{\Omega,\omega,\omega'} = \gamma^{\uparrow\uparrow}_{\Omega,\omega,\omega'} \pm \gamma^{\uparrow\downarrow}_{\Omega,\omega,\omega'}
\end{equation}
The vertices for the respected channels from Eq. \ref{eq:spin_symm} and the bubbles from Eq. \ref{eq:dual_bubble} are substituted into ladder equations:
\begin{equation}\label{eq:dual_ladder}
\Gamma^{d/m}_{\Omega,\omega,\omega'}(q) = \gamma^{d/m}_{\Omega,\omega,\omega'} + \sum_{\omega''} \gamma^{d/m}_{\Omega,\omega,\omega''} \tilde\chi_{\Omega,\omega''}(q) \Gamma^{d/m}_{\Omega'',\omega'}(q).
\end{equation}

Evaluation of Eq. \ref{eq:dual_ladder} is performed independently for each pair of vertex bosonic frequency $\Omega$ and transferred momentum $q$. 
$\gamma_{\Omega,\omega,\omega'}$ and $\Gamma_{\Omega,\omega,\omega'}(q)$ are represented as matrices in the space of fermionic Matsubara frequencies $\omega$, $\omega'$, and $\tilde\chi_{\Omega,\omega''}(q)$ is a diagonal matrix. 
In the matrix notation for each $\Omega$ and $q$ this equation reads 
\begin{equation}\label{eq:lin_alg_inv}
(\hat 1 - \hat \gamma \tilde \chi)\hat \Gamma  = \hat \gamma.
\end{equation} 
This equation is physically correct only when the maximum eigenvalue of $\hat \gamma \tilde \chi$ is smaller than one, i.e. all eigenvalues of the matrix $\hat D = \hat 1 - \hat \gamma \tilde \chi$ are positive. 
To determine it, an LU-decomposition of the matrix $\hat D$ is performed, and a determinant $D(\Omega, q)$ is calculated.
If $D(\Omega, q) > 0$, then Eq. \ref{eq:lin_alg_inv} is solved and $\Gamma$ is obtained at constant computational complexity.
When $D(\Omega, q) \leq 0$, the DF solution is outside of the convergence radius of the ladder approximation.
Given that the resulting solution is unique, one can extend this convergence radius by doing a low-order iterative evaluation of $\Gamma$ and checking if the inversion of Eq. \ref{eq:dual_ladder} can be obtained on the next DF iteration. 
 
Once the fully dressed vertex function $\Gamma_{\Omega,\omega,\omega'}$ is obtained, it is used in the Schwinger-Dyson equation to obtain the fermionic self-energy $\tilde \Sigma_{\omega, k}$. 
The equation reads:
\begin{equation}\label{eq:sd}
\tilde \Sigma_{\omega, k} = \frac{T}{2 N_k^D}  \sum_{\Omega, q} \left( 3 \left[\Gamma^m_{\Omega,\omega,\omega}(q) - \frac{1}{2}\Gamma^{(2), m}_{\Omega,\omega,\omega}(q) \right] + \Gamma^d_{\Omega,\omega,\omega}(q) - \frac{1}{2}\Gamma^{(2), d}_{\Omega,\omega,\omega}(q)  \right) \tilde G_{\omega, k + q},
\end{equation}
where $\Gamma^{(2)} = \hat \gamma \tilde \chi \hat \gamma $ indicates the second order (first iteration) correction from Eq. \ref{eq:dual_ladder} to avoid diagrammatic overcounting. 

The resulting self-energy is used to feed back the dual Green's function through the Dyson's equation:
\begin{equation}\label{eq:dyson}
\tilde G^{-1}_{\omega k} = \left[G^{(0}_{\omega k}\right]^{-1} - \tilde \Sigma_{\omega k}
\end{equation}

The procedure is repeated until convergence of $\tilde G$ is achieved. 

\subsection{Resulting observables}
Once the fully converged dual Green's function $\tilde G$, and consequently, self-energy $\tilde \Sigma$, vertices $\Gamma^{d/m}$ are obtained, they can be used in determining the lattice correlators. Specifically, 
\begin{itemize}
\item Lattice self-energy: 
\begin{equation}\label{eq:sigma_lat}
\Sigma_{\omega, k} = \frac{\tilde \Sigma_{\omega, k}}{1 - g_\omega \Sigma_{\omega, k}} + \Sigma^{DMFT}_{\omega},
\end{equation}
where $\Sigma^{DMFT}_{\omega} = i\omega + \mu - \Delta_{\omega} - g_\omega^{-1}$. 
\item Lattice Green's function
\begin{equation}\label{eq:glat}
G_{\omega,k} = \left[\Delta_{\omega} - \varepsilon_{k}\right]^{-1} + \left[\Delta_{\omega} - \varepsilon_{k}\right]^{-1} g_{\omega}^{-1} \tilde G_{\omega, k} g_{\omega}^{-1} \left[\Delta_{\omega} - \varepsilon_{k}\right]^{-1}.
\end{equation}
Eqs. \ref{eq:sigma_lat} and \ref{eq:glat} are related through a Dyson's equation for $G$ and $\Sigma$. 
\item Charge/Spin susceptibility 
\begin{align}
\chi^{\mathrm{ch/sp}} (\Omega, q) & = -T \sum_{\omega k} G_{\omega k} G_{\omega+\Omega k+q} + \sum_{\omega,\omega'} L_{\Omega, \omega}(q) \Gamma^{d/m}_{\Omega,\omega,\omega'}(q) L_{\Omega, \omega'}(q), \\
L_{\Omega, \omega}(q) & = -T \sum_k \mathfrak{G}_{\omega,k} \mathfrak{G}_{\omega + \Omega,k + q} \\
\mathfrak{G}_{\omega,k} & = \tilde G_{\omega k} \frac{\tilde G^{(0)}_{\omega, k} + g_\omega}{\tilde G^{(0)}_{\omega, k}}
\end{align}
\end{itemize}

\section{Implementation}\label{sec:imp}
The code is distributed as a C++ library with compiled executables \texttt{hub\_df\_cubic{\bf D}d}, where \texttt{\bf D} labels the number of dimensions ($D=1,2,3,4$). We use the opensource \texttt{gftools} library \cite{gftools} for algebraic operations with single- and multi-particle Green's functions and it's interface to the \texttt{ALPSCore} libraries \cite{ALPSCore} for loading/saving \texttt{hdf5} objects. Calculation of bubbles of Green's functions is performed using FFT transformations. The loop through the Brilloin zone in Eq. \ref{eq:sd} is optimized by sampling only the irreducible wedge and reweighting contributions from different symmetry points. The code is available on \url{https://github.com/aeantipov/opendf}.

The library distribution allows for flexibility with the input data. In order to use provided \texttt{hub\_df\_cubic{\bf D}d} executables the input data should be stored in the \texttt{hdf5} file, under any chosen group name. It should contain the following objects, loadable by \texttt{gftools} routines. 
\begin{itemize}
\item \texttt{/gw0} and \texttt{/gw1}. $g_{\omega \uparrow}$ and $g_{\omega \downarrow}$.  Assumed to be identical.
\item \texttt{/delta0} and \texttt{/delta1}, storing $\Delta_\omega$.
\item \texttt{/F00} and \texttt{/F01}. $\gamma^{\uparrow\uparrow}$ and $\gamma^{\uparrow\downarrow}$ respectively.
\item Other parameters, like $\mu$ and hopping unit are provided through the provided command line interface.
\end{itemize}
The output of the code is stored in a generated or existing hdf5 file under the \texttt{/df} section.

\section{Performance analysis}\label{sec:perf}
To illustrate the performance of the code, we provide an example input generator for the Hubbard model at $U \gg t$ (the case of  ``atomic limit'') and particle-hole symmetry. In this case 
\begin{align}\label{eq:atomic_limit1}
g_{\omega} & = \frac{1}{2}\left[\frac{1}{i\omega - U/2} + \frac{1}{i\omega + U/2}\right] \\
\Delta_{\omega} & = 2Dg_{\omega} \label{eq:atomic_limit2} \\ 
\gamma^{\uparrow\uparrow}_{\Omega,\omega,\omega'} & = \frac{\beta U^2}{4} (\delta_{\omega_1,\omega_2} - \delta_{\omega_1,\omega_4} ) \Lambda_{\omega_1}\Lambda_{\omega_3} \label{eq:atomic_limit3} \\ 
\label{eq:atomic_limit4}
\gamma^{\uparrow\downarrow}_{\Omega,\omega,\omega'} & = -U + 
\frac{U^3}{8}\frac{\omega_1^2 + \omega_2^2 + \omega_3^2 + \omega_4^2}{\omega_1^2\omega_2^2\omega_3^2\omega_4^2} + \frac{3U^5}{16}\frac{1}{\omega_1\omega_2\omega_3\omega_4}  \\
& \notag + \frac{\beta U^2}{4} \frac{1}{1 + \exp(\beta U /2 )} 
(2\delta_{ \omega_2, -\omega_3} + \delta_{\omega_1, \omega_2}) 
\Lambda_{\omega_2} \Lambda_{\omega_3}  \\ 
& \notag - \frac{\beta U^2}{4} \frac{1}{1 + \exp(-\beta U /2 )} 
(2\delta_{ \omega_1, \omega_4} + \delta_{\omega_1, \omega_2}) 
\Lambda_{\omega_1} \Lambda_{\omega_3},
\end{align}
where $\Lambda_\omega = 1 + U^2/(4\omega^2)$ and fermionic notation $\omega_1 = \omega, \omega_2 = \omega + \Omega, \omega_3 = \omega' + \Omega, \omega_4 = \omega'$ are used to simplify the notation. The corresponding program is provided with the code. 

The numerical solution of dual fermion equations requires introducing several control parameters. In particular,
the vertex function $\gamma_{\Omega,\omega,\omega'}$ is sampled on a grid with a cutoff $N_\Omega$ in bosonic and $N_{\omega}$ fermionic frequencies and the Brilloin zone is sampled on a finite grid of size $N_k$, providing the total volume of the system $N=N_k^D$. We analyze the convergence of the code upon tuning $N_{\Omega}$, $N_{\omega}$ and $N_k$ and the computational scaling below. Eqs. \ref{eq:atomic_limit1}, \ref{eq:atomic_limit2}, \ref{eq:atomic_limit3}, \ref{eq:atomic_limit4} are used to provide the input to the code and the system is evaluated in $2$ dimensions, at $U=8$, $\mu = \frac{U}{2}$. We choose the value of $g= \tilde G_{i\pi / \beta, 0, 0}$ to control the convergence. After making several runs, changing control parameter $N_x$, with $x = \{ \Omega, \omega, k \}$, we extrapolate the value to $N_x \to \infty$ and plot the normalized difference 
\begin{align}\label{eqn:deltag}
\delta_{g} = \left|\frac{g_{N_x} - g_{N_x \to \infty}}{g_{N_x \to \infty}}\right|
\end{align}
to evaluate the error. The maximum run-time of this benchmark data is $\approx 4$ min.

\begin{figure}[ht]
\includegraphics[width=1.0\columnwidth]{time_bfreqs.pdf}
\caption{(a) Execution time as a function of the number of bosonic frequencies $N_{\Omega}$; (b) Systematic error $\delta_g$ of the dual fermion Green's function $\tilde G_{i\omega, k}$ at $i\omega = i\pi / \beta, k = (0,0)$ for the Hubbard model in 2 dimensions at different number of bosonic frequencies $N_b$. (c) $\delta_g(N_{\Omega})$ in a logarithmic scale. }
\label{fig:benchmark_b}
\end{figure}

Fig. \ref{fig:benchmark_b} shows the performance of the \texttt{opendf} code upon the change of the total number of bosonic frequencies in the vertex $\gamma_{\Omega}$ for a fixed number of fermionic frequencies $N_{\omega}=48$ for a $16 \times 16$ k-space grid. The computational effort, indicated by the time to convergence in Fig.~\ref{fig:benchmark_b}(a), grows linearly in $N_{\Omega}$, while the error, as defined in Eqn.~\ref{eqn:deltag} and shown in frame (b), is on the order of a few percent and decreases quadratically, indicating that this error can be eliminated with large enough $N_{\Omega}$.

\begin{figure}[ht]
\includegraphics[width=1.0\columnwidth]{time_ffreqs.pdf}
\caption{(a) Execution time as a function of the number of fermionic frequencies $N_{\omega}$; (b) Systematic error $\delta_g$ . (c) $\delta_g(N_{\omega})$ in a logarithmic scale..}
\label{fig:benchmark_f}
\end{figure}

We analyze the performance of the code with respect to the change of the total number of fermionic frequencies in Fig. \ref{fig:benchmark_f}.  In this benchmark we fix the number of bosonic frequencies, $N_{\Omega}=4$, and perform the calculation on a $16\times16$ k-space grid. The computational scaling seen in Fig.~\ref{fig:benchmark_f}(a)  grows quadratically, while the relative error shown in Fig.~\ref{fig:benchmark_f}(b) remains an order of magnitude smaller, as compared to the variation in $N_{\Omega}$ shown in Fig.\ref{fig:benchmark_b}(b). We see that $g$ converges very quickly with an increase of $N_{\omega}$ and for large values of $N_{\omega}$ the scaling stops and becomes constant.

\begin{figure}[ht]
\includegraphics[width=1.0\columnwidth]{time_kpts.pdf}
\caption{(a) Execution time as a function of Volume=$N_k^D$. (b) Systematic error $\delta_g$ as a function of the volume. (c) log-scale of (b).}
\label{fig:benchmark_kpts}
\end{figure}

The performance of the code with respect to the change of number of k-space samples within the Brilloin zone, is plotted on Fig. \ref{fig:benchmark_kpts}. The computational effort (frame (a)) scales linearly with the volume $N_k^D$ of the system and the error is very small and scales as $(N_k^D)^6$, until it reaches the floating point precision.

\section{Example - Hubbard model, 2 dimensions}\label{sec:ex}

\begin{figure}[ht]
\centering
\includegraphics[width=0.6\columnwidth]{DF_U12.eps}
\caption{Momentum Dependence}
\label{fig:sigma}
\end{figure}

\begin{figure}[ht]
\centering
\includegraphics[width=\columnwidth]{susc.pdf}
\caption{Left: Spin and charge static susceptibility at $\Omega = 0$. Right: Colormap plot of the spin-susceptibility. }
\label{fig:susc}
\end{figure}

We provide the practical illustration of the method for the Hubbard model in $D=2$ dimensions. We show the $k$-dependence of the real part of the lattice self-energy $\Sigma_{\omega k}$ at $i\omega = i\pi/\beta$ in Fig.~\ref{fig:sigma} for the case of particle-hole symmetry at $U/t=8$ and compare it with available data from the Dynamical Cluster Approximation \cite{MaierJarrell:2005}. The data from the ALPS DMFT \cite{ALPSDMFT} package with CT-AUX solver \cite{Gull2008b} was used as an input. The DMFT self-energy is momentum-independent and $\mathrm{Re} \Sigma_{\omega k}^{\mathrm{DMFT}} = 0$ and is plotted with a dashed line. Taking into account the spatially dependent corrections by the dual fermions leads to a correct momentum-dependence of the self-energy, matching in this case the DCA result. The detailed comparison between multiple methods will be discussed elsewhere [Benchmark ref]. 

We illustrate the susceptibility output of the opendf code on Fig. \ref{fig:susc}. Plotted are the static spin- and charge- susceptibilities at $U/t = 8$ for the particle-hole symmetric case. The spin susceptibility, peaked at $(\pi,\pi$) due to antiferromagnetic fluctuations is much larger, than the charge one. 

\section{Conclusion}\label{sec:conclusions}
In this contribution we have introduced the \texttt{opendf} project which contains command line tools to solve the dual fermion self consistency conditions and computes non-local corrections to the local solutions provided by DMFT.  As such, opendf fills an important void in the existing available software packages, allowing for quick implementations of the DF technique on top of existing DMFT routines.  

The code, which is available at \url{https://github.com/aeantipov/opendf} will continue to undergo development.  Goals include extending the choice of sampled diagrams, incorporating broken-symmetry phases and multiorbital systems, attempt at determining accuracy by checking the contributions of various diagrams. 

\section*{Acknowledgements}
Authors are grateful to D. Hirschmeier for fruitful discussions and acknowledge the Simons collaboration on the many-electron problem for financial support and for it's support of the \texttt{ALPSCore} project.

\bibliography{opendf}

\end{document}


